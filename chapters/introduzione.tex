\chapter*{Introduzione}                 %crea l'introduzione (un capitolo
                                       %   non numerato)
%%%%%%%%%%%%%%%%%%%%%%%%%%%%%%%%%%%%%%%%%imposta l'intestazione di pagina
\rhead[\fancyplain{}{\bfseries
INTRODUZIONE}]{\fancyplain{}{\bfseries\thepage}}
\lhead[\fancyplain{}{\bfseries\thepage}]{\fancyplain{}{\bfseries
INTRODUZIONE}}
%%%%%%%%%%%%%%%%%%%%%%%%%%%%%%%%%%%%%%%%%aggiunge la voce Introduzione
                                        %   nell'indice
\addcontentsline{toc}{chapter}{Introduzione}

La tesi è strutturata come segue:
\begin{itemize}

    \item Nel primo capitolo si presenta una breve panoramica sulla storia e sulle caratteristche principali della tecnologia blockchain. In particolare si parlerà di Bitcoin come punto di partenza della blockchain delle, e successivamente dell'arrivo di Ethereum, con la blockchain non più applicabile esclusivamente al campo delle criptovalute, ma come base per lo sviluppo di applicazioni decentralizzate, dette DApps, per i compiti più disparati. Si evidenzierà poi come le blockchain presentano problematiche che ne impediscono un'adozione su scala globale, e in particolar modo ci si concentrerà sulla mancanza di scalabilità di esse. Infine, si presenterà la piattaforma Radix DLT, un'alternativa scalabile alle blockchain classiche.
    
    \item Nel secondo capitolo si presenterà l'idea per la realizzazione di un'applicazione utilizzando gli strumenti per gli sviluppatori messi a disposizione di Radix. L'idea per l'applicazione riguarda il pagamento di royalties in campo musicale. Si vuole così verificare la fattibilità di costruire applicazioni basate sulla piattaforma Radix. Si parerà dunque dell'architettura dell'applicazione.
    
    \item Nel terzo capitolo si presenteranno le scelte progettuali seguite durante l'implementazione dell'applicazione. Si parlerà poi del funzionamento dell'applicazione. Nella conclusione si parlerà poi di quali potranno essere eventuali miglioramenti.
    
\end{itemize}

%%%%%%%%%%%%%%%%%%%%%%%%%%%%%%%%%%%%%%%%%non numera l'ultima pagina sinistra
\clearpage{\pagestyle{empty}\cleardoublepage}