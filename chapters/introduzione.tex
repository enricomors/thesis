\chapter*{Introduzione}                 %crea l'introduzione (un capitolo
                                       %   non numerato)
%%%%%%%%%%%%%%%%%%%%%%%%%%%%%%%%%%%%%%%%%imposta l'intestazione di pagina
\rhead[\fancyplain{}{\bfseries
INTRODUZIONE}]{\fancyplain{}{\bfseries\thepage}}
\lhead[\fancyplain{}{\bfseries\thepage}]{\fancyplain{}{\bfseries
INTRODUZIONE}}
%%%%%%%%%%%%%%%%%%%%%%%%%%%%%%%%%%%%%%%%%aggiunge la voce Introduzione
                                        %   nell'indice
\addcontentsline{toc}{chapter}{Introduzione}

La tecnologia Blockchain, introdotta inizialmente come registro distribuito per le transazioni della moneta digitale Bitcoin nel 2008, ha suscitato grande interesse nel corso degli ultimi anni. In poco tempo si è capito come tale tecnologia potesse essere applicata ai campi più disparati, e non soltanto alle monete elettroniche. Fra le tante applicazioni che si prospettano per le blockchain vi sono, ad esempio, il settore energetico, il sistema sanitario, e la tracciabilità delle catene di fornitura di beni, servizi o prodotti alimentari. Il futuro insomma, sembra orientato verso un'adozione delle blockchian su scala globale. Tuttavia, è necessario risolvere una serie di problematiche affinchè ciò possa avvenire. Il difetto forse più importante da questo punto di vista, è che la tecnologia blockchain, per come è stata progettata, non è scalabile. Non sarebbe pertanto capace di supportare un volume di transazioni pari o superiore a quello di circuiti di pagamento globali come Visa o MasterCard, per fare un esempio. Mentre vengono studiate possibili soluzioni per aggirare il problema nelle blockchain esistenti, sono nate piattaforme alternative alle blockchain che promettono una maggiore scalabilità, nonchè facilità nel costruire applicazioni basate su esse. E' il caso di Radix DLT. In questa tesi ho deciso di concentrarmi su Radix DLT e di realizzare una piccola applicazione per provare con mano gli strumenti di sviluppo a disposizione. 

La tesi è strutturata come segue:
\begin{itemize}

    \item Nel primo capitolo si presenta una breve panoramica sulla storia e sulle caratteristche principali della tecnologia blockchain. In particolare si parlerà di Bitcoin come punto di partenza della blockchain delle, e successivamente dell'arrivo di Ethereum, con la blockchain non più applicabile esclusivamente al campo delle criptovalute, ma come base per lo sviluppo di applicazioni decentralizzate, dette DApps, per i compiti più disparati. Si evidenzierà poi come le blockchain presentano problematiche che ne impediscono un'adozione su scala globale, e in particolar modo ci si concentrerà sulla mancanza di scalabilità di esse. Infine, si presenterà la piattaforma Radix DLT, un'alternativa scalabile alle blockchain classiche.
    
    \item Nel secondo capitolo si presenterà l'idea per la realizzazione di un'applicazione utilizzando la libreria JavaScript messa a disposizione degli sviluppatori dal team di Radix. L'idea per l'applicazione riguarda il pagamento di royalties in campo musicale. Si vuole così verificare la fattibilità di costruire applicazioni basate sulla piattaforma Radix. Si parerà dunque dell'architettura dell'applicazione.
    
    \item Nel terzo capitolo si presenteranno le scelte progettuali seguite durante l'implementazione dell'applicazione. Si parlerà poi del funzionamento dell'applicazione. Nella conclusione si parlerà poi di quali potranno essere eventuali miglioramenti.
    
\end{itemize}

%%%%%%%%%%%%%%%%%%%%%%%%%%%%%%%%%%%%%%%%%non numera l'ultima pagina sinistra
\clearpage{\pagestyle{empty}\cleardoublepage}